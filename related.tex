\section{Related Work}
\label{sec:related}

%We discuss prior work on different threat models, goals,
%architectures, and strategies for mitigating network side channels.
%We discuss prior work along three axes: shaping strategy,
%system architecture, and threat model and goals.

\textbf{Traffic shaping for web.}
Prior work used traffic shaping~for defending against website
fingerprinting attacks.
%These strategies can be classified into two types: static and dynamic.
%
Walkie-Talkie \cite{wang2017walkietalkie}, Supersequence
\noindent \cite{wang2014supersequence}, and Glove \cite{nithyanand2014glove} use
clustering techniques to group objects of a corpus and then shape the
traffic of all objects within each cluster to conform to a similar pattern.
Traffic morphing \cite{wright2009morphing} makes the traffic of one page look
like that of another.
These techniques compute traffic shapes that envelope or resemble the network
traces of individual objects. Hence, they require a large number of
traces to account for network variations.
{\sys} dynamically adapts traffic shapes based on the
prevailing network~conditions.
%while conforming with DP guarantees
%and, thus is, computationally less expensive.

%BuFLO~\cite{dyer2012peekaboo},
Cs-BuFLO~\cite{cai2014csbuflo}, Tamaraw~\cite{cai2014tamaraw}, and
DynaFlow~\cite{lu2018dynaflow}, determine traffic shape directly at runtime.
%and FRONT and GLUE~\cite{gong2020zero} are
%dynamic techniques that determine traffic shape at runtime.
They pad object sizes to values that are correlated
with the original object sizes, such as the next
multiple or power of a configurable constant.
%The constants determine
%%the padding overhead and
%the number of objects padded to the same size.
These defenses provide privacy akin to k-anonymity, but have no control on the
the size of the anonymous cluster.
Tamaraw\cite{cai2014tamaraw} formalizes the privacy guarantees.
{\sys}'s DP guarantees are strictly stronger than Tamaraw's (proof
in~\S\ref{appendix:tamaraw}).
%\todo{{\sys} provides a stronger privacy guarantee based on DP (see
%\S\ref{appendix:tamaraw} for a proof).}
%Tamaraw provides a mathematical model to quantify the privacy guarantees of
%fingerprinting defenses.
%
%We prove that their notion of privacy is strictly weaker than differential
%privacy for traffic streams (see \Cref{appendix:tamaraw}).
%FRONT~\cite{gong2020zero} only obfuscates the beginning of the network
%traces of web pages.
%and the gap between consecutive traces, respectively.
%by adding random amount of dummy packets.
%BuFLO and Cs-BuFLO transmit fixed-sized packets at constant rate over a
%predefined duration of time. GLUE is specifically designed to obfuscate the
%beginning and end of a web trace by introducing additional packets between
%traces originating from different webpages.

%Despite extensive profiling, these techniques are unlikely to be able to cover
%all network conditions at runtime.
%{\sys}, on the other hand, is a dynamic traffic shaping solution, which
%provides strong, quantifiable differential privacy guarantees on both packet
%sizes and timing.

\textbf{Differential privacy over streams.}
%\textbf{Differential Privacy.}
%Kellaris et al.~\cite{kellaris2014differentially} also study DP over streams.
%They introduce \textit{$w$-event} privacy based on a neighboring definition that
%only allows stream to differ over {\em one} window $w$.
Dwork et al.~\cite{dwork2010dpcontinual} study DP on streams, in which
neighboring streams differ in at most one element and the query
counts over the stream prefix (without forgetting
old information) under a fixed DP budget for the entire stream.
%is constantly updated over time
%This paper focues on a different setting than ours, in which neighboring streams
%differing in only one element, but the DP query (here a counter) is constantly
%updated over time without forgetting past information (the counter counts the
%whole prefix of the stream), under a fixed DP budget for the entire stream.
{\sys} requires a stronger neighboring definition to model application data
streams (Def. \ref{def:neighboring-streams}), but is able to forget the past by
dropping stale data from our queue (\Cref{assumption:window}) and compose
privacy loss over time.

{\sys}'s neighboring definition is closer to that of user-level DP over streams
in \cite{dwork2010pan}. Instead of counting discrete change,
however, we use the L1 distance which enables coarsening.
Pan-privacy considers an adversary that can compromise the internals of
the algorithms (\eg our buffering queue). This makes the design of algorithms
challenging and costly.
%(the paper provides several impossibility results).
Instead, we consider the buffering queue private and focus on
practical algorithms to study the privacy/overheads trade-off.

%\ml{[I we want to cite pan-privacy] Another work \todo{[cite Pan-privacy]}
%considers user-level DP over streams, which allows for larger changes and is
%closer to our own neighbouring definition (although we use the L1 distance which
%enables coarsening, not counts of discrete change). This paper considers a
%stronger threat model, called PAN-privacy, where an attacker can compromise the
%internals of the algorithms (e.g. our buffering queue). This makes the design of
%algorithms much more challenging and costly (the paper provides several
%impossibility results). Instead, we consider the buffering queue private and
%focus on practical algorithms to study the privacy/utility trade-off.}

Kellaris et al.~\cite{kellaris2014differentially} introduce a notion of DP,
called $w$-event privacy, for streams of length $w$.
%Each event in a stream is~a query result on a static database state at a
%specific timestamp.
Neighboring stream pairs are those whose individual events are pairwise
neighbors within a window of upto $w$.
%Neighboring streams are otherwise identical.
%This is not a practical neighboring definition for the
%streams we consider, and would provide weak privacy.
%{\sys}'s ``database'' consists of video streams and its
{\sys}'s neighboring definition accounts for the maximum stream distance over
{\em any} window of length $\winlen$, which is a better match for the streams
we consider.
%{\sys} also uses the L1-norm distance between streams, whereas $w$-event privacy
%considers the number of differences for discrete elements.

%\ml{Not doing it as I'm not sure if there are other reasons, but I suggest
%    moving this one last of this section.}
%In a similar work,
Zhang et al.~\cite{zhang2019statistical} generate differentially private shapes
for video streams using Fourier Perturbation Algorithm (FPA)
\cite{rastogi2010differentially}.
FPA transforms a finite time series of bursts, into a series of DP shaped bursts
of the same length.
FPA requires the entire stream's profile upfront, and cannot guarantee complete
transmission of an input stream within the shaped trace.
%Unlike {\sys}'s dynamic shaping mechanism, however, FPA requires predefined
%time series associated with traffic traces in advance.
{\sys}'s DP guarantees simply compose over burst interval sequences, thus
allowing shaping of streams of arbitrary lengths with
quantifiable privacy and overheads.
%{\sys} does not constrain transmission duration; rather it can provide DP
%guarantees composed over burst interval sequences of arbitrary lengths.
%{\sys} dynamically shapes bursts in periodic intervals; the DP guarantees
%compose over burst-interval sequences of arbitrary lengths.
%Moreover, it is not guaranteed that the FPA algorithm's output will have
%sufficient capacity (in terms of number of bytes) to accommodate the input
%traffic trace. Authors refer to this issue as deficit phenomena.




%\paragraph{Mitigations for VoIP.}

%\paragraph{Network side-channel mitigation in IoT traffic.}

%\if 0
%%%%%%%%%%%%%%%%%%%%%%%%%%%%%%%%%%
%% remove due to space constraints
%% point covered in intro
%%%%%%%%%%%%%%%%%%%%%%%%%%%%%%%%%%
\textbf{Adversarial defenses.}
%Given the recent advancements in machine learning-based traffic analysis
%attacks, new defensive mechanisms based on adversarial examples have emerged.
%Mockingbird~\cite{rahman2020mockingbird} and Dolos~\cite{shan2021dolos} attempt
%to defeat existing classifiers using adversarial inputs.
%Dolos~\cite{shan2021dolos} computes input-agnostic adversarial patches of dummy
%packets.
%Mockingbird~\cite{rahman2020mockingbird} uses an optimization problem to make
%one stream look like another by adding adversarial inputs to the input stream.
Adversarial defenses~\cite{rahman2020mockingbird, hou2020wf, abusnaina2020dfd,
shan2021dolos, nasr2021defeating, gong2022surakav}
generate targeted and low-overhead noise to defeat specific classifiers.
%attempt to defeat existing classifiers with small amounts of carefully-crafted
%perturbations in the input streams.
%The targeted noise incurs less overhead but cannot defeat another classifier.
%However, the adversarial examples generated for one classifier may not
%generalize to another classifier.
%However, an adversary can train its classifier with adversarial
%samples to become resilient to defenses.
%Thus, this approach leads to an arms race between attackers and defenders.
{\sys} provides provable and configurable privacy against both SOTA as
well as future classifiers.
%\fi


\if 0
%%%%%%%%%%%%%%%%%%%%%%%%%%%%%%%%
%% skip due to space constraints
%%%%%%%%%%%%%%%%%%%%%%%%%%%%%%%%
\paragraph{Mitigations for contention-based network side channels.}
An adversary may be colocated with a victim in the same network and may share
links with a victim’s traffic (e.g., a tenant VM colocated on a Cloud server or
rack). In such a scenario, the adversary can induce contention with the victim’s
traffic on a shared link and use the resulting variations in its own traffic
shape to infer the victim’s traffic shape and ultimately the victim’s secrets.
Such leaks can be mitigated using TDMA
scheduling on the shared links \cite{vattikonda2012tdma, beams2021ifs},
%or by physically isolating the network of the tenants,
which eliminates the
adversary’s ability to observe the victim’s traffic. {\sys} focuses
on an orthogonal threat model where the adversary is privileged and owns (a
part) of the victim’s network path (\eg ISP). It may be infeasible for a tenant
to achieve isolation from the adversary (ISP); thus, {\sys} leverages traffic
shaping to protect the victim’s traffic in the face of adversarial observations.
\fi

\textbf{Network side-channel mitigation systems.}
%\textbf{Pacer.}
%Pacer \cite{mehta2022pacer} proposes the idea of a cloaked tunnel, which is
%conceptually similar to {\sys}’s shaping tunnel.
{\sys}'s shaping tunnel is conceptually similar to Pacer's~\cite{mehta2022pacer}
cloaked tunnel.
%However, they are similarity ends there.
%have different threat models.
%First,
However, Pacer mitigates leaks of a Cloud tenant's secrets to a
colocated adversary through contention at shared network links.
%Thus, Pacer's tunnel endpoint must be integrated with end hosts.
Pacer's cloaked tunnel controls the transmit time of
TCP packets in accordance with the shaping schedule and congestion control
signals. Thus, Pacer requires
non-trivial changes to the network stack on the end hosts.
%First, Pacer’s focus is on an adversary colocated with an application on the end
%host, which infers the application’s traffic shape through indirect
%contention-based side channels.
%Secondly, Pacer implements pacing below the transport layer, which forces it to
%implement congestion awareness in the pacing component. Therefore, Pacer’s
%tunnel endpoint is tightly integrated with end hosts and requires non-trivial
%changes to the network stack on the end hosts.
%
%In {\sys} protects the traffic of applications from an adversary on the public
%Internet.
{\sys} protects applications that are behind private networks but communicate
using the public Internet.
{\sys}'s tunnel endpoints can be placed at the interface of
the private-public networks, \eg in a middlebox, thus supporting
multiple applications without modifying end hosts.
Moreover, by shaping above the transport layer, {\sys} needs to control only the
precise timing for generation of bursts of DP length and not the subsequent
%packetization and
transmission to the network.
%Thus, {\sys}'s design is more modular and portable than Pacer's.

%Besides the tunnel architecture, the
%The shaping strategies of Pacer and {\sys} differ as well.
%Pacer splits a server's dataset into clusters of a minimum size $k$, while
%minimizing padding overhead for the dataset. For each cluster, Pacer then
%computes a traffic shape based on network traces of all cluster objects.
%Pacer's clustering may not suffice to protect objects within the clusters
%against an adversary with information about the popularity of objects within the
%cluster.
%{\sys}'s DP shaping guarantees privacy even against a strong adversary with
%auxiliary information.
%\todo{{\sys}'s overheads are comparable to Pacer's.}
%Pacer assumes that the shape of clients' requests to services is non-sensitive
%and thus shapes only the server responses. Furthermore, Pacer clusters a
%server's dataset into clusters of a minimum size k, while minimizing padding
%overhead for the dataset. These assumptions and optimizations may not suffice
%when an adversary has auxiliary information, such as the popularity of
%individual objects during different times of the day. {\sys}’s DP shaping hides
%the timing as well as the content of the traffic even from a strong adversary
%with access to auxiliary information.
%\todo{Despite the stronger privacy guarantees, {\sys}’s overheads are comparable
%to Pacer's.} {\as{does that mean our guarantees are stronger than pacer? in
%that case I don't think it is correct.}}

Ditto~\cite{meier2022ditto} and {\sys} propose shaping traffic at network nodes
separate from end hosts.
%Ditto~\cite{meier2022ditto} implements shaping in programmable switches, while
{\sys} proposes a hardware-independent, modular and portable
middlebox architecture that can be integrated with end hosts, routers,
gateways, or even programmable switches as in Ditto.
%{\sys}'s middlebox architecture is hardware-independent, modular and portable,
%and can be integrated with either end hosts or intermediate nodes, such as
%routers, or gateways.

%\todo{
%%Ditto defines a sequence of packet sizes and send packets at fixed
%%interval regardless of the size and timing of application packets.
%%It pads application packets to ensure that their sizes follow the predefined
%%sequence and buffers them to maintain the fixed packet interval.
%Unlike Ditto, {\sys} dynamically shapes traffic based on application traffic
%while maintaining DP guarantees.
%}

%\textbf{Censorship circumvention proxies.}
\textbf{Systems with other goals.}
%Tor \cite{Tor}, FTE \cite{dyer2013protocol}, Scramblesuit
%\cite{winter2013scramblesuit}, Skypemorph \cite{mohajeri2012skypemorph}, Balboa
%\cite{rosen2021balboa}, DeltaShaper \cite{barradas2017deltashaper}, and Protozoa
%\cite{barradas2020poking} are
Censorship circumvention systems~\cite{mohajeri2012skypemorph,
winter2013scramblesuit, barradas2017deltashaper,
barradas2020poking, rosen2021balboa}
rely on traffic obfuscation, scrambling, and transformations of a sensitive
application’s shape to that of a non-sensitive application. These techniques
prevent identification of original protocols by deep packet inspection,
but do not prevent inference of secrets from traffic shapes.
%Unlike {\sys}, they do not prevent inference of secrets from traffic shapes.
%They rely on Pluggable Transports (PTs), which is a SOCKS proxy-based
%architecture. PTs run on application end hosts or on trusted nodes in the Tor
%network and communicate over HTTP(S).
%In contrast, {\sys} ensures privacy of the traffic content from an adversary.
%{\sys}'s design is modular and portable.
%For instance, it can integrate a SOCKS proxy in the {\ushaper} process, although,
%its own transport-layer proxy can accept arbitrary application protocols.

\if 0
%%%%%%%%%%%%%%%%%%%%%%%%%%%%%%%%%%%%%
%% remove due to space constraints
%% points covered in intro and design
%%%%%%%%%%%%%%%%%%%%%%%%%%%%%%%%%%%%%
QCSD~\cite{smith2022qcsd} is a client-side only defense framework for website
fingerprinting attacks. QCSD instructs the server about sending payload or dummy
data using flow control signals in the headers of the QUIC packets sent from the
client. Moreover, it uses QUIC's PADDING frames to transmit dummy data.
Unlike the STREAM frames, PADDING frames do not generate
acknowledgements \cite{rfc9000} and, thus, are distinguishable on the network.
Overall, QCSD is a best-effort approach and cannot provide {\sys}'s guarantees.
%To shape the traffic from the server to the client, it leverages the per-stream
%flow control of QUIC to determine when to send actual data, dummy data, or a
%combination of both.
%It is important to note that QCSD is a best-effort approach and does not
%provide a guarantee that the traffic shaping will perfectly match the defense
%mechanism in either directions. Therefore, the QCSD framework is not
%appropriate for emulating the traffic shaping mechanism of {\sys}.
\fi

\if 0
\paragraph{Limitations of QUIC features.}
An alternative design approach is to utilize QUIC padding frames instead of
introducing a separate dummy stream to the connection. However, as stated in the
QUIC specification~\cite{rfc9000}, padding frames are not acknowledged by the
receiver. This poses a significant challenge for us since we require all events
observable by potential adversaries, including ACK messages, to be independent
of the content of streams.
MASQUE (Multiplexed Application Substrate over QUIC Encryption) is a framework
that facilitates the simultaneous execution of multiple networking applications
within an HTTP/3 connection and enables a QUIC client to negotiate proxying
capabilities with an HTTP/3 server.
While MASQUE currently lacks a stable specification and implementation, it holds
potential for providing negotiation capabilities for {\sys} in the future.
\fi

%\textbf{Metadata private communication.}
%\textbf{Anonymous communication.}
Karaoke~\cite{lazar2018karaoke} and Vuvuzela~\cite{van2015vuvuzela} are
anonymous messaging systems that use~DP to hide participants in a conversation,
but use constant-rate traffic among the participants.
\update{
AnoA~\cite{backes2013anoa} is a framework to analyze anonymity
properties of anonymous communication
protocols. AnoA supports DP based quantification for various
properties, such as sender anonymity and sender unlinkability.
%AnoA~\cite{backes2013anoa} is an anonymity analysis framework that establishes
%quantifiable and comparable anonymity properties, taking inspiration from
%privacy frameworks such as differential privacy.
%AnoA provides metrics to quantitatively assess the anonymity provided by an
%Anonymous Communication Protocol.
}
%differential privacy to hide  participants  in a message
%conversation, but rely on constant traffic shaping  among the participants.
{\sys}'s differentially-private traffic shaping hides the traffic content. In
principle, {\sys} could be combined with an anonymous communication system to
provide both content privacy and anonymity with~DP.

%\todo
%{
%\paragraph{Other Methods (To Add).}
%\begin{enumerate}
%\item DFD~\cite{abusnaina2020dfd}: Stochastically extends the size of traffic
%bursts to confuse the classifier.
%\item WF-GAN~\cite{hou2020wf}: Using a Generator/Discriminator to add
%adversarial noise to traffic such that the classifier missclassifies the input.
%\item Mockingbird~\cite{rahman2020mockingbird}: Molds traffic sequence to match
%Adversarial Example sequences generated against an effective Deep Learning
%classifier.
%\item Dolos~\cite{shan2021real}:injects dummy packets into traffic traces by
%computing input-agnostic adversarial patches that disrupt deep learning
%classifiers used in WF attacks. Patches are applied in real time and
%parameterized by a user-side secret, ensuring that attackers cannot use
%adversarial training to defeat Dolos.
%\end{enumerate}
%}
