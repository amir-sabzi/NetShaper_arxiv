\section{Overview}
\label{sec:overview}


%Protecting clients' privacy requires ensuring confidentiality of their network
%traffic content.
%The content may be leaked through the traffic shape, \ie, the packet sizes,
%inter-packet intervals, and the direction of the traffic. Additionally, it may
%be leaked through the time of a client's initiation of communication with a
%service, and the duration of the communication.
%\todo{In applications where the traffic passes through multiple nodes on the
%Internet, the content may additionally be correlated with the sequence of
%interactions with the nodes, which constitute a traffic DAG (directed acyclic
%graph).}

%\am{Describe a real-world motivating example.}

A secure and practical {\nsca} mitigation system must satisfy the following
design goals:
{\bf G1.} Address leaks through all aspects of the shape of transmitted traffic,
{\bf G2.} Provide quantifiable and tunable privacy guarantees for the
communication parties,
{\bf G3.} Minimize overheads incurred while guaranteeing privacy,
{\bf G4.} Support a broad class of applications, and
{\bf G5.} Require minimal changes to the applications.
%{\bf G5.} Enable tuning between privacy guarantees and overheads with minimal
%changes modifications to the applications. {\as{Maybe we can combine G5 and G3}}

{\sys}'s shaping prevents leaks of the traffic content through sizes, timing,
and direction of packets \todo{along all communication links between application
nodes} (G1).
%\todo{It can also hide the precise times of network activity.} (G1).
%
We present the key ideas of {\sys}'s shaping that help achieve the remaining
goals. We then present {\sys}'s threat model.
%We present the key ideas of our solution that help achieve the above goals and
%the threat model and assumptions we are operating on.
%We start by listing the key goals for designing a {\nsca} mitigation system. We
%then describe the threat model and assumptions we are operating on, and finally
%the key ideas of our solution.

%\subsection{Design goals}
%\label{subsec:goals}
\subsection{Key ideas}
\label{subsec:key-ideas}

\paragraph{Differentially private shaping.}
%Ensuring confidentiality of content requires shaping the
%packet sizes and timing along all links involved in a client's network traffic.

%A strawman approach is constant shaping, which would require continuously
%transmitting fixed-length packets with a fixed inter-packet interval.
%Constant shaping is a secure approach, but can induce significant bandwidth and
%latency overheads. Prior work has proposed ways to allow variations in traffic
%shaping that reduce overheads at the cost of privacy leaks.
%\todo
As discussed in \S\ref{sec:intro}, unlike constant shaping and
variable-but-static shaping, dynamic shaping offers the most flexibility for
adapting traffic shape at runtime based on workload patterns and, therefore, can
reduce significantly reduce overheads.
%Dynamic shaping can help to reduce costs of mitigating network side channels at
%the cost of some privacy leak.
{Unfortunately, existing dynamic shaping techniques either have unbounded privacy
leaks or offer only weak privacy guarantees.
%Moreover, some of these techniques require extensive profiling of an
%application's network traffic traces.
}
%\am{Check recent techniques.} \am{How can we compare privacy
%leakage of k-anonymity vs differential privacy?}{\as{I think k-anonymity has
%$\varepsilon$ of infinite as the output probability for any two trace from two
%different groups of anonymity is not bounded.}}
%
%{\sys} instead relies on a dynamic shaping technique that provides
%%allows variations in traffic shapes while providing
%%\todo
%{quantifiable privacy leaks with tunable bounds} and without requiring
%extensive pre-profiling of application traffic.
%%\todo{{\sys}'s shaping hides the time of initiation and termination of
%%transmissions}, as well as the content of the transmissions (G1).
%Specifically, {\sys} provides differential privacy guarantees in its shaping
%(G2).
%Specifically, it generates differentially-private burst length for transmission
%in periodic intervals.
{\sys}'s \todo{novel} differential privacy (DP) based shaping strategy provides
quantifiable privacy leaks with tunable bounds and without relying on profiling
of application traffic (G2).
\todo{The DP guarantees are determined on fixed-length burst intervals and
composed across consecutive intervals to quantify the overall privacy
guarantees for an application.}
%\am{\ml{@Mathias: verify if the above English explanation of the DP guarantees
%is accurate.}}
%
%{\sys}'s shaping incurs lower overheads compared to constant shaping as well as
%other prior techniques (G3).
%We discuss the shaping technique in \S\ref{sec:dp} and evaluate its performance
%in \S\ref{sec:eval}.

\paragraph{Shaping in a middlebox.}
%\S\ref{fig:minesvpn-overview} provides an overview of {\sys}'s architecture.
{\sys} relies on the abstraction of a tunnel for implementing traffic shaping.
Application traffic is shaped between the tunnel endpoints such that an
adversary observing the tunnel traffic cannot infer application secrets.
In principle, a tunnel endpoint could be integrated with the end host hosting an
application (\eg in a VM isolated from the end-host application), or in a
separate node through which the application's traffic passes.
{\sys} relies on the second approach and implements the tunnel endpoint as a
middlebox, which could be integrated with an existing network element, such as a
router, a VPN gateway, or a firewall.
%(see \S\ref{sec:implementation} for more details).
The middlebox implementation enables securing multiple applications without
requiring modifications on individual end hosts (G4).
Furthermore, for multiple flows that require the same tunnel, it allows pooling
the flows under the same privacy guarantees, which helps to amortize the
per-flow overhead (G3).
%We elaborate on the {\sys}'s implementation in \S\ref{sec:implementation} and
%evaluate the performance optimization in \S\ref{sec:eval}.
%\ml{Here we need to emphasize that sharing a pipe is very good for privacy
%guarantees (assuming we evaluate it? To me that's a core benefit of the
%middlebox deisgn so it'd be really good to have it.) Maybe something like
%``Importantly, our middlebox design enables pooling multiple clients through
%the same protected tunnel, providing privacy guarantees to mulitiple
%connections at a fixed overhead (\S\ref{sec:dp}).''}

\paragraph{Minimal modifications to end applications.}
%\todo{TODO.}
By default, {\sys} shapes all traffic through a tunnel with a fixed differential
privacy guarantee, \ie with a single value of $\epsilon$ and $\delta$. However,
an application can explicitly specify its privacy requirements ($\epsilon$ and
$\delta$), as well as
bandwidth and latency constraints, and any prioritization preferences on a
per-flow basis to its local middlebox. This requires small modifications in the
application to transmit a shaping configuration message to the middlebox along
with the application flow. Thus, {\sys} offers a balance between being fully
application-agnostic versus optimizing for privacy or overhead with minimal
support from the applications (G2, G5).
%\am{We need to know how {\sys} should use DP parameters for 5s intervals or for
%sensitivity of 2MB.}

\subsection{Threat Model}
\label{subsec:threat-model}
%{\sys}'s goal is to protect an application's secrets from being leaked to a
%network adversary.
%We assume that the communicating nodes are inside a trusted private network,
%\eg behind a VPN gateway node, which an adversary cannot compromise.
%\todo{We assume that all the application endpoints are non-malicious and do not
%explicitly leak the secrets themselves. The secrets could be leaked only through
%side channels. A service serving both privacy-sensitive and privacy-insensitive
%clients is partitioned into two instances, with each instance dedicated to
%serving clients with similar privacy requirements. For the privacy-sensitive
%clients, we assume the corresponding service endpoint agrees to serve the
%clients according to their privacy requirements.}

%{\sys}’s goal is to protect an application’s secrets from being leaked to a
%network adversary.
{\sys}'s goal is to hide the content of an application's network
traffic.
%\todo{as well as the time of initiation and the duration of the network
%activity}.
%\am{Despite the ``always-on'' tunnels, we still have coarse-grained leaks of
%start times of flows when a flow selects different DP parameters than default.
%If we further imagine starting tunnels just before transmissions, such as for a
%multi-hop scenario, we should not talk about hiding initiation times.}
We assume that the communicating nodes are inside separate trusted private
networks (\eg each node is behind a VPN gateway node), which an adversary
cannot compromise. We assume that all the application endpoints are
non-malicious and do not leak the secrets themselves.
%%%%%%%%%%%%%%%%%%%%%%%%%%%%%%%%%%%%%%%
%% can handle non-sensitive traffic simultaneously by just using TDMA
%%%%%%%%%%%%%%%%%%%%%%%%%%%%%%%%%%%%%%%
%Furthermore, we assume that applications transmitting only non-sensitive
%traffic are partitioned from the applications transmitting sensitive traffic.
%For instance, the non-sensitive applications use a different router.

%We assume that the communication between all nodes in the client's traffic DAG
%is encrypted.
The adversary controls network links in the public Internet (\eg ISPs) and can
record, measure, and tamper with the victim's traffic as it traverses through the
links under the adversary's control.
In particular, it can precisely record the traffic shape---the sizes, timing,
and direction of packets---on all links in the victim's traffic
\todo{communication graph}.
Furthermore, the adversary can drop, replay, or inject packets into the victim's
traffic.
However, it cannot break standard cryptography, cannot compromise the victim's
VPN, or cannot impersonate its clients or servers to directly interact with the
victim \update{(thus, no covert attacks).}
\update{We do not consider threats due to observing the IP addresses of packets,
although {\sys} can hide IP addresses of application end hosts behind a shared
traffic shaping tunnel.}
%We also do not consider the threat where an adversary colocates with
%the victim inside the victim's private network to observe the victim's traffic
%shape through indirect contention signals~\cite{mehta2022pacer}.
We also do not consider the threat where a victim accidentally installs a
malicious script in the browser, thus enabling an adversary to colocate with the
victim's application and observe its
traffic~\cite{schuster2017beautyburst,mehta2022pacer}.  This is a reasonable assumption,
since a colocated adversary can exploit many other direct or indirect channels
for data leaks that will be far more efficient than
{\nsc}s~\cite{kocher2018spectre, yarom2014flushreload, liu2015llcpractical, irazoqui2015ssa, vila2017loophole}.
%\eg by renting a VM or by installing a malicious script from a malicious website
%enticing the victim to accidentally install malicious scripts from a malicious
%website.
%\todo{TODO: how do we handle victim accidentally visiting malicious websites and
%installing malicious javascript, etc. on its device? In other words, how do we
%completely rule out colocated adversary?}

%\todo{TODO: Describe the assumptions on adversary knowledge or capability w.r.t.
%differential privacy.}

In this paper, we focus on an implementation of {\sys}'s traffic shaping tunnel
within a middlebox that can be placed in front of the gateway router of an
organization.
%We discuss implementation alternatives in \S\ref{sec:discussion}.
%
{\sys}'s trusted computing base (TCB) includes all components in the
organization's private network and the middleboxes. We assume
that the middleboxes do not directly leak application secrets to the adversary.
In practice, bugs or vulnerabilities in the middleboxes could be eliminated
using standard techniques, such as software fault isolation~\cite{tan2017sfi}. We
assume the cryptographic libraries are side-channel
free~\cite{almeida2016verifying}.
%\todo{{\sys} prevents leaks of applications' secrets via side channels within its
%    middleboxes, such as timing and microarchitectural side channels (see
%    \S\ref{sec:implementation}).}

{\sys} does not address leaks of one application's sensitive data through the
traffic shape of colocated applications transmitting only non-sensitive traffic.
Such leaks may arise, for instance, due to microarchitectural interference among
the applications colocated on a host or among their flows if they pass through
shared links. Mitigating such leaks would require physically isolating the
applications and their flows. For instance, a service serving both
privacy-sensitive and privacy-insensitive clients may be partitioned into two
instances, with each instance dedicated to serving clients with similar privacy
requirements. For the privacy-sensitive clients, we assume the corresponding
service instance agrees to serve the clients according to their privacy
requirements. Alternatively, end hosts could implement mitigations for various
side channels to support colocated applications \cite{mehta2022pacer,
page05partitionedcache, shi2011limiting, kim2012stealthmem,
varadarajan2014scheduler, braun2015robust}. To mitigate leaks among colocated flows at shared
links, {\sys} assumes that the network paths of the sensitive and non-sensitive
flows are completely partitioned. In practice, this limitation can be removed by
combining {\sys}’s traffic shaping with TDMA scheduling \cite{beams2021ifs,
vattikonda2012tdma}.

{\sys} does not prevent leaks of applications' secrets via side channels, \eg
microarchitectural side channels, in its middleboxes. We discuss potential
mitigations in \S\ref{subsec:impl-security}.

\update{Under these assumptions, {\sys} prevents leaks of application secrets
%\todo{and the timing of network activity}
through the sizes and the delays between the network packets transmitted in
either direction between the application endpoints.}
%\todo{Although anonymity is not the focus of our work, {\sys} hides the number
%and identity of the participants communicating between two sites.}

%\update{Under these assumptions, {\sys} prevents leaks of application secrets
%through the sizes of the network packets, the inter-packet intervals, the time
%of initiation of a network activity, or the duration of network activity in
%either direction of communication between the application endpoints.}
