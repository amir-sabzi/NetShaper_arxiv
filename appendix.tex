\section{Proof of {\sys}'s DP based Shaping}
\label{appendix:dp}
\begin{proposition*}
  {$\sys$} enforces $\Delta \leq D_s$.
\end{proposition*}
To prove the proposition stated above, we will first establish the following lemma.
\begin{lemma}
Assume two neighboring traffic streams, $S_t$ and $S_t'$ ($\|S_t - S_t'\|_1 \leq D_s$), transmitted through a
middlebox with the algorithm \ref{alg:middle-box-all} as the shaping mechanism.
If they both reshaped to the same traffic stream, $O$, then,  at any given time
$\tau$, the queue states for the first and second streams are $D$-close.
In other words, we have:

\begin{equation}\label{equ:composition}
        \forall \tau > 0 : |Q_{\tau} - Q_{\tau}'| \leq D_s
\end{equation}
\end{lemma}

\begin{proof}
The mechanism of algorithm \ref{alg:middle-box-all} dequeues the data from
the queue periodically at time $t=lT$, where $l \in \mathbb{N}$.
Thus, the queue state of stream $S$ at any given time, $\tau$,
is a function of three variables:
\begin{enumerate}
        \item The queue state in the last round of algorithm execution ($Q_{\lfloor \frac{\tau}{T} \rfloor -1}$).
        \item The amount of traffic has been dequeued from the queue ($\sum_{(\lfloor \frac{\tau}{T} \rfloor  - 1)T \leq t < (\lfloor \frac{\tau}{T} \rfloor)T} P^S_t$).
        \item The amount of real traffic received as a part of stream after last
        execution round ($D^{R}_{\lfloor \frac{\tau}{T} \rfloor}$).
\end{enumerate}
Therefore, for the queue state after dequeue process we have:
\begin{equation}
                \label{equ:queue-state}
                Q_{\lfloor \frac{\tau}{T} \rfloor} =
                {Q_{\lfloor \frac{\tau}{T} \rfloor -1}}
                +
                {\sum_{(\lfloor \frac{\tau}{T} \rfloor  - 1)T \leq t < (\lfloor \frac{\tau}{T} \rfloor)T} P^S_t}
                -
                {D^{R}_{\lfloor \frac{\tau}{T} \rfloor}}
\end{equation}
\\
The equation \ref{equ:queue-state} can be described as a recursive equation
w.r.t to execution rounds.
If we have $\lfloor \frac{\tau}{T} \rfloor = i$, then:
\begin{align}\label{equ:queue_state_expansion}
        Q_{i} - Q_{i}'
        =
        (Q_{i-1} - Q_{i-1}')
        +
        \\ \nonumber
        (\sum_{(i - 1)T \leq t < iT}P^S_t
        -
        \sum_{(i - 1)T \leq t < iT}P^{S'}_t)
        -
        (D^{R}_{i} - D^{R'}_{i})
\end{align}
We know both streams, $S_t$ and $S_t'$ have reshaped to the same output, $O$.
However, the amount of traffic has dequeued from the queue for the first stream
might not be the same as second stream.
The reason is that one queue might not have enough traffic to satisfy
the DP decision, which in turn, leads to padding.
Nevertheless, based the algorithm \ref*{alg:middle-box-all}, DP mechanism always
satisfy the following inequality:
\begin{equation}\label{equ:queue-dequeue}
        (Q_{i} - Q_{i}').(D^{R}_{i} - D^{R'}_{i}) \geq 0
\end{equation}
As we mentioned before, the $Q_i$ determines the size of the queue after $D_i^S$
bytes of data are dequeued from it.

To show why always inequality \ref{equ:queue-dequeue} holds, we consider all
three possible scenarios for two queue states after one round of algorithm
\ref*{alg:middle-box-all} execution:
\begin{enumerate}
    \item $Q_i,Q_i' > 0$: This means there are still data resided in both
    queues.  We also know that the output stream for both queues are the same,
    $O_i$. This means no padding have been applied to either of two queues.
    Then: $D^{R}_{i} = D^{R'}_{i} \rightarrow D^{R}_{i} - D^{R'}_{i} = 0$.
    \item $Q_i,Q_i' = 0$: This simply implies $Q_{i} - Q_{i}'=0$
    \item $Q_i = 0, Q_i'>0$: The output stream, $O_i$, is the same for both
    queues. Thus, the first queue provided less real data since it has emptied
    out. This means: $D^{R}_{i} \leq D^{R'}_{i}$.
    \item $Q_i' = 0, Q_i>0$: With the same of justification of previous case we
    have:
    $D^{R}_{i} \geq D^{R'}_{i}$
\end{enumerate}
Now, using the inequality \ref{equ:queue-dequeue}, we want to prove the
following:
\begin{align}
        |Q_{i} - Q_{i}'|
        \leq
        |Q_{i-1} - Q_{i-1}'|
        +
        \sum_{(i - 1)T \leq t < iT}|P^S_t - P^{S'}_t|
\end{align}

Without loss of generality, assume $(Q_{i}-Q_{i}') \geq 0$.
% \footnote{The proof is similar for the case of $(Q_{i}-Q_{i}') \leq 0$.}
We can re-write the equation \ref*{equ:queue_state_expansion} as following:
\begin{align*}
        \nonumber
        |Q_{i} - Q_{i}'|
        =
        Q_{i} - Q_{i}'
        =
         (Q_{i-1} - Q_{i-1}')
        +
        \\
        (\sum_{(i - 1)T \leq t < iT}P^S_t - P^{S'}_t)
        -
        (D^{R}_{i} - D^{R'}_{i})
        \\
        \leq
        (Q_{i-1} - Q_{i-1}')
        +
        (\sum_{(i - 1)T \leq t < iT}P^S_t - P^{S'}_t)
        \\
        \leq
        |
        (Q_{i-1} - Q_{i-1}')
        +
        (\sum_{(i - 1)T \leq t < iT}P^S_t - P^{S'}_t)
        |
        \\
        \leq
        |Q_{i-1} - Q_{i-1}'|
        +
        \sum_{(i - 1)T \leq t < iT}|P^S_t - P^{S'}_t|
\end{align*}
Intuitively, this means the dequeue stage, never increases the difference
between two queues.
\\
With $d_i = |Q_{i}-Q_{i}'|$ and $d_0 = 0$, we have:
\begin{align}
        \nonumber
        d_i
        \leq
        d_{i-1}
        +
        \sum_{(i - 1)T \leq t < iT}|P^S_t - P^{S'}_t|
        =
        \\ \nonumber
        0 + \sum_{j=0}^{i}\big({\sum_{(j - 1)T \leq t < jT}|P^S_t - P^{S'}_t|}\big)
        \leq
        \\
        \sum_{0 \leq t < iT} |P^S_t - P^{S'}_t|
        =
        \|S_t - S_t'\|_1
        \leq
        D_s
\end{align}
Thus, the proof is completed.
\end{proof}
Therefore, the maximum difference between queue sizes (i.e. the sensitivity, $\Delta$) is always bounded by $D_s$.
\begin{align}
  \Delta = \max_{\tau}\max_{S, S'} | Q_{\tau} - Q'_{\tau} | \leq D_s
\end{align}
