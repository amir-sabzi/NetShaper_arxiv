\subsection{{Security Analysis}}
\label{subsec:impl-security}
{\sys} provides the following security property: an adversary cannot infer
application secrets from observing tunnel traffic. This
property is ensured by a combination of a secure shaping strategy, the tunnel design,
and implementation.

{\bf S1. Secure shaping strategy.} The tunnel transmits traffic in
differentially private-sized bursts
in fixed intervals. Thus, the overall shape is DP. The proof of DP is in
\S\ref{appendix:dp}.

{\bf S2. Secure tunnel design.}
(i) The privacy guarantees of a tunnel are configured before the start of
application transmission and do not change during the tunnel's lifetime.
(ii)~The tunnel mediates control between the end hosts, \eg by
transmitting custom connection establishment and termination messages. These
messages are subject to the same DP shaping as the payload traffic
(\S\ref{subsec:design-overview}).
%(iii)~All payload traffic of a sensitive application passes through a tunnel
%between a pair of middleboxes and is subjected to DP shaping and encryption.
(iii) The payload and dummy bytes in network packets are indistinguishable
because all payload and dummy bytes are packaged into QUIC packets and
encrypted uniformly. Moreover, QUIC handles acknowledgements, congestion
control, and loss recovery for both payload and dummy bytes uniformly
(\S\ref{subsec:design-overview}).

{\bf S3. Secure middlebox implementation.}
{(i)} The unshaped traffic between an end host and its local middlebox is
not visible to an adversary.
(ii) {\dshaper} follows the tunnel design in transmitting payload and dummy
bytes.
%
%(i) The control traffic between end hosts is intercepted at the middleboxes and
%is not directly sent into the tunnel.
%
%(ii) The payload and dummy bytes in network packets are indistinguishable. This
%is because both payload and dummy bytes are packaged into QUIC packets and
%encrypted uniformly. Thus, QUIC handles acknowledgements, congestion control,
%and loss recovery for both payload and dummy bytes
%(\S\ref{subsec:impl-shaping}).
%(iii) The sizes and timing of transmitted packets are secret-independent,
%because
(iii) The time required for {\prepare} to prepare and enqueue shaped
buffers is masked to secret-independent times. The packetization of~buffers in
QUIC is secret-independent~(\S\ref{sec:implementation}) and
%(\S\ref{subsec:impl-shaping-security}) and
thus retains DP guarantees after post-processing.
Any delays in transmitting the buffers can arise only due to congestion or
packet losses in the tunnel network, which are secret-independent events.
%(iii) The time required to {\em prepare} the DP-sized buffers is independent of
%application secrets. This is because {\sys}'s periodic interval is configured
%such that it can handle all DP sizes. Thus, any delays in transmitting the
%buffers can only arise due to congestion or packet losses in the tunnel network,
%which are secret-independent events.
%\todo{(iv) TODO: Verify that the middleboxes' handling of flow control do
%not reveal secrets.}
%
%
%{\bf S6.} No leaks are possible via the traffic shape of a non-sensitive
%application, which is physically isolated (by assumption in our threat model).

%\am{Discuss how middlebox's acks, handling of cc, flow control, loss recovery do
%not reveal secrets.}

%\subsection{Mitigating side-channel leaks}
%\label{subsec:impl-side-channels}
%A key challenge in ensuring sound {\nsca} mitigation could be side channels that
%may impact the middlebox execution and reveal secrets. We first list the
%sensitive information that can be potentially leaked via side channels and then
%discuss mitigations for the leaks.
%
%%\paragraph{Secrets.}
%There are three main types of secrets: the application data, the session keys
%for E2E encryption on the end hosts, and the session keys for tunnel encryption
%in the middlebox. Deviations of the tunnel transmissions from the DP schedule
%could be correlated with these secrets, thus revealing the secrets to a network
%observer.
%%%
%Since a middlebox is physically isolated from the end hosts, application keys
%and data cannot directly influence the middlebox execution. However, the
%application secrets could manifest into its traffic shape through its flow
%control mechanism. We elaborate this with two scenarios.
%%Thus, we only need to {\em performance isolate} the middlebox
%%execution from the application's flow control and from the tunnel's keys.
%
%\emph{Application flow control.}
%Consider an end-host application transmitting two secrets files A and B, each of
%length $L_A$ and $L_B$ bytes, respectively, with $L_A < L_B$. Suppose that the
%application transmits $L_A$ bytes to the middlebox in time $T_A < T$, whereas it
%transmits $L_B$ bytes in time $T_B$ that spans two intervals of duration $T$,
%say $L_{B_1}$ and $L_{B_2}$ in each interval, such that $L_{B_1} + L_{B_2} =
%L_B$.
%%Consider two consecutive intervals at the middlebox $T_i$ and $T_{i+1}$ and
%For simplicity, suppose that the {\DPlogic} suggests transmitting $X$ bytes in
%each shaping interval, where $X > max(L_A,~L_{B_1},~L_{B_2})$.
%Now, suppose the application transmits files A and B consecutively, with file A
%being transmitted just before the interval $T_i$ and file B being transmitted
%just before the interval $T_{i+2}$ at the
%middlebox. The {\DPlogic} will prepare buffers for transmission in each interval
%as follows: $T_i: [L_A,~X - L_A],~T_{i+1}: [0,~X],~T_{i+2}: [L_{B_1},~X -
%L_{B_1}],~T_{i+3}: [L_{B_2},~X - L_{B_2}]$.
%%The first buffer will contain $L_A$ bytes of A and $X - L_A$ dummy
%%bytes, while the second buffer will entirely contains $X$ dummy bytes.
%First, processing payload and dummy bytes could take different amounts of time.
%For
%instance, adding dummy bytes could consistently take a few microseconds longer
%than copying payload bytes.
%Secondly, the execution of the {\DPlogic} could be influenced by the presence or
%absence of payload traffic from the application. For instance, in a shaping
%interval, the {\DPlogic} could be delayed or interrupted if the application
%sends traffic and otherwise not. Such variances in the execution of {\DPlogic}
%could ultimately manifest into observable timing of the buffer transmissions in
%the tunnel, which could reveal information about the buffer contents (\eg
%whether they contain dummy bytes or not).
%
%\emph{DShaper secret keys.}
%The final secret in the tunnel are the keys used to encrypt the shaped traffic
%in each tunnel session. Prior work has shown how crypto keys can be leaked via
%various timing and microarchitectural side channels. Leak of the session keys
%would allow a network observer to decrypt all tunnel traffic, thus invalidating
%packet padding mitigations.
%
%Thus, we need to {\em performance isolate} the middlebox execution from the
%application's flow control and from the tunnel's keys.
%
%\paragraph{Mitigation.}
%For performance isolation, we rely on a combination of resource partitioning,
%performance masking, and \todo{constant-time implementation} techniques.
%First, we isolate the host-facing network stack and the tunnel stack into
%separate VMs, which run on separate cores and have dedicated access to separate
%physical NICs, DRAM and cache partitions. Furthermore, the interrupts for each
%VM are dedicated to the respective cores.
%
%Next, the execution of the {\shapedVM} could be affected by the memory partition
%hosting the datastructures shared with the {\unshapedVM} as well as the tunnel's
%keys which are used for encrypting the shaped packet.  To address this, we pin
%the datastructures in cache and implement the {\DPlogic} in constant time, and
%we assume that the crypto implementation is constant time as well.
%Finally, we {\em mask} the variance in the execution time of the path from the
%{\DPlogic} to the \todo{tunnel's network stack}. For this, we profile the
%worst-case execution time of this path and we delay forwarding of each packet
%to the
%stack until this worst-case time. Note this masking technique is conceptually
%similar to that proposed in Pacer~\cite{mehta2022pacer}. \am{refine the
%description}
%
%With the above mitigations in place, the only remaining side channels are due to
%the resources that fundamentally remain shared -- the internal buses and the
%off-core units that are shared among all cores (\eg APIC).  \todo{{\sys} does
%    not explicitly address side channel leaks via these resources.} \am{Check
%    DAGuise to justify this.}

\subsection{Deployment and Maintenance}
\label{subsec:design-discussion}
{\sys}'s tunnel endpoint design and implementation are both very modular and
portable.
The middlebox components are compatible with all application layer protocols
(\eg HTTPS, QUIC-TLS) and network stacks (\eg TCP, UDP, QUIC stacks).
The {\ushaper} could also be implemented as a standard SOCKS5 proxy~\cite{torpt}.
%The {\ushaper} component is compatible with all application layer protocols,
%such as HTTPS and QUIC-TLS. It could also be implemented as a standard SOCKS5
%proxy~\cite{torpt}. Similarly, the {\dshaper} is compatible with all transport
%protocols and network stacks.

A tunnel endpoint could be integrated with any node along an application's
network path as long as the application traffic is unobservable until egress
from the tunnel.
%
By integrating with a trusted VPN gateway of an organization,
network administrators could manage ``long-term'' tunnels between the
organization's campuses and support multiple applications without modifying
individual end hosts. For instance, separate tunnels may be configured
per-application according to the organizational needs, or
configurations may be adapted based on coarse-grained changes in the traffic
patterns through the~day.

Alternatively, by integrating with end user devices, \eg with VPN clients,
users could instantiate a new bidirectional tunnel with a service before
each network activity, choose a different configuration for each tunnel
instance, and close the tunnel after completion of the activity. A key
requirement would be to secure the tunnel endpoint's execution from any internal
side channels on the end host, which would be now more prevalent than in the
middlebox setup.

