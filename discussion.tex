\section{Limitations and Discussion}
\label{sec:discussion}

We discuss extensions of the {\sys}'s prototype to support additional usages.

\if 0
\paragraph{Multiplexing with non-sensitive flows}
%{\sys} currently assumes that the network paths of non-sensitive and sensitive
%flows are physically isolated. To overcome this limitation, {\sys} can integrate
To route non-sensitive flows alongside sensitive flows, {\sys} can integrate
TDMA (time division multiple access) scheduling \cite{vattikonda2012tdma,
beams2021ifs} in its architecture. Specifically, {\sys} must employ a
synchronous TDMA schedule on its ingress and egress ports and on its CPUs for
each traffic class. When the slot for non-sensitive flows is scheduled, the
traffic is simply routed from the {\sys}’s ingress port to the egress port and
vice versa without any modifications.
%
Note that each tunnel endpoint can use a different TDMA schedule. A synchronous
TDMA schedule between each pair of endpoints may be useful for minimizing
overheads but is not essential for end-to-end security guarantees.
\fi
%\subsection{Alternate deployments of {\sys}}

%\if 0
\textbf{Scheduling multiple tunnels.}
%When handling multiple tunnels, {\sys} must ensure that the
%unshaped traffic of one tunnel is not leaked to another tunnel. Such leaks may
%arise via contention side channels in the mdidlebox.
When handling multiple tunnels, the unshaped traffic of one tunnel could
indirectly affect the shape of traffic in another tunnel, thus potentially
leaking the traffic of the first tunnels.
%For this,
{\sys} must isolate the tunnels from each other in the middlebox.
%Based on \S\ref{sec:implementation}, we require that the {\prepare}
%and QUIC worker threads must be performance isolated from {\ushaper} in each
%tunnel, (ii) the timing of {\prepare} must be masked to secret-independent
%times.
{\sys} could partition the middlebox cores into three groups, each core group
hosting the {\ushaper} process, the {\prepare} threads, and the QUIC worker
threads from different tunnels.
Additionally, {\sys} must use a TDMA schedule among the {\prepare} threads,
while \update{padding} each thread's execution to a secret-independent time.
%In {\sys}, only the {\dshaper} components need to be isolated. {\sys} uses a
%TDMA to schedule the {\dshaper} processes of different tunnels.
Since each {\prepare} thread enqueues shaped buffers at secret-independent
times, QUIC workers of different tunnels can be scheduled to transmit
packets arbitrarily.
%can subsequently %package the buffers into packets and
%transmit the packets in their tunnels following any arbitrary schedule.
%
%{Determining optimal TDMA schedules and their adaptation to the changing
%number of active tunnels is left to future work.}
%\am{Not implemented, remove?}
%\fi

\textbf{Generalization to multi-node communication.}
%\todo{TODO: single application with multiple tiers (e.g., CDNs to serve
%    different parts of a web request) -- general DAG topology}
%\todo{TODO: multi-party communication (e.g., a group VoIP call relayed through a
%    central server) -- star topology}
%{\sys} mitigates network side channel leaks in bidirectional communications
%between a client-server pair. However, {\sys}'s DP shaping can be extended
Some applications may involve multiple nodes forming a communication graph,
where the communication graph could uniquely identify application secrets.
For example, dynamic web pages may be recognized from the traffic generated to
all servers hosting web page components.
%serving dynamic web pages may require downloading embedded objects from
%multiple  servers.
{\sys} can be extended to protect secrets in such applications.
{\sys}'s middlebox is portable and can be easily added in front
of each node to shape traffic on all links.
A client could establish tunnels among all pairs of nodes in an application for
the duration of its interaction with the application.
Each link could use its own DP parameters for shaping; the privacy
guarantees over the communication graph would be a composition of DP guarantees
of all links in the graph.
We leave a detailed analyses of privacy guarantees and overheads in a multi-node
application to future work.
%The key requirement is to model a ``buffering queue'' to define the maximum
%information that could be accessed by an adversary in the communication graph.

\textbf{Integrating {\sys} with end hosts.}
To ensure end-to-end security while accessing sensitive services from outside
organization premises, users could use a VPN client on their devices that is
enhanced with {\sys}'s shaping capabilities. A key challenge would be to ensure
performance isolation of the {\sys}-enhanced VPN client from the application
handling sensitive data \cite{mehta2022pacer}.

\textbf{Integrating {\sys} with other technologies.}
{\sys} can be considered as a network function (NF) and can be integrated with a
chain of other NFs, such as load balancers, firewalls, and intrusion-detection
systems. {\sys} could be the first function in an NF chain, interfacing with the
external network, which forwards inbound flows to other NFs after dropping
padding and shapes only select outbound flows forwarded by the next NF in the
chain. Placing {\sys} at the beginning of the chain would minimize the
communication bandwidth within the NF chain and the private network.
%\am{just mention as a one line somewhere earlier?}
%\paragraph{Interaction between {\sys} and DPI technologies.}
%{\sys} does not conflict with deep packet inspection (DPI) technologies, which
%an organization may use to monitor their inbound traffic for suspicious
%behaviors. However, {\sys} protects against third-party DPI which may


%\subsection{Generalization for orthogonal goals}
%\paragraph{Anonymity.}
