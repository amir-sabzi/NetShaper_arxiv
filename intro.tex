\section{Introduction}
\label{sec:intro}

%The rising demands of remote work and multi-site connectivity have led to an
%increase in usage of VPN services. While VPNs encrypt network traffic to protect
%against data breaches, they are insufficient to protect against {\em {\nsca}
%attacks}. In a {\nsca} attack, an {\em adversary} infers a {\em victim's}
%secrets through observations of one or more features of network traffic, {\eg}
%packet sizes, timings, and direction, which constitute the victim's {\em traffic
%shape}. Prior work has shown that traffic shape can reveal rich information
%about a user's network activity; it can reveal VoIP chat content~\cite{voip},
%video streams~\cite{schuster2017beautyburst}, and even users' medical or financial secrets
%from traffic on websites~\cite{webreality}.
With the proliferation of TLS and VPN, traffic encryption has become the de
facto standard for securing data in transit in Internet applications. Traffic
can be encrypted at various layers, such as HTTPS, QUIC, and IPSec.
%which can protect the traffic content from network adversaries.
%
While these protocols prevent {\em direct} data breaches on the Internet, they
cannot~prevent leaks through {\em indirect} observations
of the encrypted traffic.

Indeed, encryption cannot conceal the shape of an
application’s traffic, \ie the sizes, timing, and number of packets sent
and received by an application. In many applications, these parameters strongly
correlate with sensitive information. For instance, %prior work has shown that
traffic shape can reveal video
streams \cite{schuster2017beautyburst}, website visits
\cite{wang2014supersequence, bhat2019varcnn},
the content of VoIP conversations \cite{white2011phonotactic}, and
even users’ medical and financial secrets \cite{chen2010reality}.

In such {\em network side-channel leaks}, an adversary (\eg a malicious or
a compromised ISP) observes the shape of an application's traffic as it
passes through a link under its control and
infers the application's sensitive data from this shape.

%While an adversary may need to overcome practical challenges in the traffic
%measurements, such as noise due to congestion or path variations, recent ML
%advancements have greatly improved the ability to make inferences even from
%noisy data \cite{schuster2017beautyburst, bhat2019varcnn, hayes2016kfp,
%sirinam2018df}. In fact, we also present a new classifier based on temporal
%convolution networks (TCN) \cite{bai2018tcn}, which improves the state of the
%art in attack classifiers \cite{schuster2017beautyburst} and can infer video
%streams even from short bursts of noisy measurements over the Internet
%\todo{(Section X)}.

%Obfuscation techniques that rely on adding ad hoc noise in an adversary's
%measurements \cite{shan2021dolos, rahman2020mockingbird, nasr2021defeating} do not
%provide comprehensive protection against {\nsca} attacks.
%Since an adversary could observe the victim’s traffic entirely passively,
%detection of such attacks can be extremely difficult.
Obfuscation techniques, which add ad hoc noise~\cite{luo2011httpos}
or adversarial noise \cite{shan2021dolos, nasr2021defeating, rahman2020mockingbird,
hou2020wf, abusnaina2020dfd, gong2022surakav}
in an application's network traces, do not provide comprehensive protection
against {\nsca} attacks \cite{zhang2019statistical}.
In fact, recent advances in machine learning (ML)
have greatly improved the ability to filter
out noise due to congestion or path variations and infer secrets from noisy data
\cite{schuster2017beautyburst, bhat2019varcnn, hayes2016kfp, sirinam2018df}. For
instance, our own novel classifier based on Temporal Convolution
Networks (TCN) \cite{bai2018tcn} can infer video streams even from short bursts
of noisy
measurements over the Internet (see \S\ref{sec:background}~for details).
%Alternatively, the victim application could evade an adversary by
Alternatively, a sensitive application could try to improve side-channel
resilience by splitting traffic over multiple network paths
\cite{cadena2020trafficsliver, wang2022leveraging} or by
using dedicated physical links that are not controlled by the adversary.
However, such solutions are inadequate against a powerful adversary
that can monitor a large fraction of the Internet
\cite{beckerle2022splitting} and may incur prohibitive network
administration costs for small users on the Internet.
%However, such isolation needs to be guaranteed end-to-end in the victim’s path,
%which is expensive and impractical, particularly for small tenants on the
%Internet and in the presence of a potent adversary, such as an ISP.

%Obfuscation techniques rely on adding ad-hoc noise in the form of chaff or
%random delays \cite{shan2021dolos, rahman2020mockingbird, nasr2021defeating} into an adversary’s
%observations of a victim’s traffic.
%Such solutions do not provide comprehensive protection against a determined
%adversary who infers secrets even from noisy observations.

In contrast, a principled and practical approach to mitigating network
side-channel leaks is {\em traffic shaping}. It involves modifying the victim’s
packet sizes and timing to make the resultant shape independent of secrets, so
that an adversary cannot infer the secrets despite observing the (shaped)
traffic.

Constant shaping involves sending fixed-sized packets at a constant rate, which
is secure but incurs non-trivial bandwidth and/or latency overhead for
applications with variable or bursty workloads \cite{saponas2007devices}.
Variable shaping strategies attempt to adapt traffic shapes to reduce the
overhead
at the cost of some privacy. However, the state-of-the-art (SOTA) variable
shaping strategies rely on ad hoc heuristics that yield weak privacy guarantees
\cite{wang2014supersequence, nithyanand2014glove, wang2017walkietalkie} or
unbounded privacy leaks \cite{gong2020zero, cai2014csbuflo,
lu2018dynaflow, juarez2016wtfpad, cai2014tamaraw}.
%Moreover, some strategies
Some techniques provide strong guarantees but
require extensive a priori profiling of an applications' traffic to compute
shapes
%and do not adapt to real-time traffic conditions, thus incurring unnecessary
%overhead
\cite{mehta2022pacer, zhang2019statistical}.

%However, na\"ive traffic shaping strategies can be extremely inefficient in
%practice or not secure at times.
%
%
%Constant shaping involves sending fixed-sized packets at a constant rate, which
%is secure but incurs non-trivial bandwidth and/or latency overhead for
%applications with variable or bursty workloads \cite{saponas2007devices}.
%Variable shaping strategies attempt to adapt traffic shapes to reduce overheads
%at the cost of some privacy.
%These strategies can be categorized into two types: static strategies, which
%cluster traffic and precompute shapes for each cluster offline
%\cite{wang2014supersequence, nithyanand2014glove, wang2017walkietalkie,
%gong2020zero, mehta2022pacer}, and dynamic strategies, which decide the traffic
%shape at runtime based on the incident workload (\eg rounding lengths of
%individual data transfers to the next power of two \cite{cai2014csbuflo,
%lu2018dynaflow, juarez2016wtfpad, cai2014tamaraw}).
%The static strategies require extensive a priori profiling of the applications
%and are, therefore, harder to adapt to real-time traffic conditions, while the
%dynamic techniques known till date do not guarantee privacy for all data
%transfers.
%
In addition to a shaping strategy, network side-channel mitigation also
requires a robust implementation of packet padding and transmit scheduling.
Many solutions attempt to protect traffic by controlling shaping from only one
end of a communication (\ie either a client or a server) and provide only
best-effort protection
%restricted to single applications
\cite{luo2011httpos, smith2022qcsd, cherubin2017llama}. Other solutions rely on
trusting third-party mediators (\eg Tor bridges), which implement shaping
between the clients and mediators and between the servers and mediators
\cite{mohajeri2012skypemorph, winter2013scramblesuit}.
%or do not provide end-to-end security \cite{beams2021ifs}.
In Pacer \cite{mehta2022pacer}, both application endpoints integrate a
shaping system to comprehensively mitigate network side channels. However, Pacer
encumbers application end hosts with non-trivial changes in the network
stack to implement shaping, thus deterring adoption.
%Thus, most of these systems suffer from practical challenges.
%\am{needs work.}

In this work, we address two main questions. First, is there a variable traffic
shaping strategy that provides quantifiable and tunable privacy guarantees at
runtime without requiring extensive pre-profiling of application traffic?
Second, can traffic shaping be provided as a generic, portable, and efficient
solution that can be integrated in different network settings and can support
diverse applications?

We present {\sys}, a network side-channel mitigation system that answers both
questions in the affirmative.
%First, {\sys} provides a differential privacy (DP) based traffic shaping
%strategy, which is application-agnostic by default, but can adapt privacy
%guarantees and resulting overheads at runtime based on the prevailing workload
%and congestion conditions.
First, {\sys} relies on a differential privacy (DP) based traffic shaping
strategy, which provides quantifiable and tunable privacy guarantees.
{\sys} specifies the privacy parameters for a configurable window of
transmission. Moreover, it can configure the parameters independently for each
direction of traffic on a communication link. The DP guarantees can be composed
based on these parameters to achieve bounded privacy leaks for arbitrary
bidirectional traffic.
Overall, applications can tune the shaping based only on the privacy guarantees
they desire and the overheads they can afford, without the need for
profiling their traffic.
%{\sys}'s shaping strategy is application-agnostic by default, but can also adapt
%the privacy guarantees and the resulting overheads at runtime based on
%prevailing workload and congestion conditions.
%{\sys}'s theoretical DP guarantees are far stronger than those required
%for SOTA attacks, which can be defeated with even small amounts of DP noise in
%the traffic.
While strong privacy guarantees require DP parameters that incur
large overheads, in practice, {\sys} can defeat SOTA attacks with even small
amounts of DP noise, thus incurring low overheads.
%Thus,
%%Moreover, while {\sys}'s shaping strategy is application-agnostic by default,
%%\todo{that subsume the privacy guarantees of prior work \cite{cai2014tamaraw}}.
%%\ml{what is the prior work ref related to?}\ml{do we want to say quantifyable?}
%%Thus, while {\sys}'s shaping strategy is application-agnostic by default,
%applications can optionally tune the shaping for the desired privacy
%guarantees and the affordable overheads.
%\am{confusing; reword}
%\update{Our shaping provides strong mathematical guarantees for privacy, which
%subsume the privacy guarantees of prior work.}

Secondly, we present a traffic shaping tunnel with a modular endpoint design
that can conceptually be integrated with any network stack and within any node.
The tunnel implements padding and transmit scheduling of packets while
adhering to the DP guarantees {\em by design}.
%Furthermore, {\sys} ensures that packet sizes and timing are not affected by
%internal side channels, thus securely eliminating all network side-channel leaks
%{\em by design}.
{By placing the tunnel endpoint in a middlebox at the edge of the
private network, {\sys} can simultaneously protect
the traffic of multiple applications. Moreover, the middlebox can amortize the
shaping overheads among multiple flows without compromising the privacy for
individual flows.
}

Together, the DP shaping strategy and {\sys}'s tunnel provide effective {\nsca}
mitigation for diverse applications, such as video streaming and web services,
with modest overheads.
To the best of our knowledge, {\sys} is the first system to provide dynamic
traffic shaping with quantifiable and tunable privacy guarantees based on DP.
%Together with the tunnel, {\sys} provide effective {\nsca} mitigation for
%diverse applications, such as video streaming and web services, with modest
%overheads.

%Unlike prior work, {\sys} can shape the traffic of
%applications without requiring any profiling of their network traces.
%Furthermore, we present a middlebox-based implementation of {\sys} that can
%support shape traffic of multiple clients and servers without requiring
%modifications on individual end hosts.
%\todo{We believe that {\sys} can eliminate the arms race in {\nsca} attacks and
%defenses and can provide a portable and configurable framework for deploying
%mitigations for applications with diverse traffic characteristics and in
%different settings.}
%\todo{TODO: quantifiable implies being able to tune the tradeoff}

\textbf{Contributions.}
(i)
%We demonstrate the viability of network side-channel attacks on video
%streams over the Internet using a SOTA CNN-based classifier, namely the ``Beauty
%and the Burst'' (BB) classifier
%\cite{schuster2017beautyburst}.
We design a new attack classifier based on a
Temporal Convolution Network (TCN) \cite{bai2018tcn}
%that is robust than the BB classifier
and demonstrate its ability to infer
videos streams from traffic shapes under noisy network traffic measurements in
the Internet (\S\ref{sec:background}).
(ii) We model network side-channel mitigations as a differential privacy problem
and provide a traffic shaping strategy that offers
($\epsilon$,$\delta$)-differential privacy guarantees (\S\ref{sec:dp}).
%(ii) We present a proxy-style architecture for shaping an application's traffic
%(iii) We design a traffic shaping tunnel that intercepts and shapes application
%traffic and present a security analysis of the design (\S\ref{sec:design}).
%(iv) We design and develop an instance of the tunnel using QUIC-based
%middleboxes, which supports shaping along with congestion control, flow
%aggregation and prioritization (\S\ref{sec:implementation}).
(iii) We design a QUIC-based traffic shaping tunnel and present a
middlebox-based implementation of the tunnel, which supports traffic shaping
while adhering to DP guarantees (\S\ref{sec:design}).
%(iii) Based on the architecture, we present the design and implementation of a
%traffic shaping system using QUIC-based middleboxes and an analysis of the
%security of the design and the implementation (\S\ref{sec:implementation}).
%(iv) We develop a new attack classifier based on a temporal convolution network
%(TCN) \cite{bai2018tcn} that is more robust to noises in network traffic measurements
%compared to the state-of-the-art CNN-based classifier
%\ml{cite. Also a TCN is a kind of CNN, so this sounds a bit weird. We could
%remove CNN, or focus on the core differences (finer grain inputs?)}
%and we demonstrate that
%{\sys}'s shaping of video traffic makes it secure against even the stronger
%classifier (\S\ref{sec:eval}).
(iv) We demonstrate {\sys}'s efficacy in defeating a SOTA classifier
\cite{schuster2017beautyburst} and our new TCN classifier.
%(\S\ref{sec:eval}).
%(v)
We empirically evaluate the tradeoffs between {\sys}'s privacy guarantees
and performance overheads while mitigating {\nsca} leaks in two classes of
applications that have already been used in prior work, namely video streaming
and web service (\S\ref{sec:eval}).
%\todo{Result overview.}
(vi) We present a formal proof of {\sys}'s differential privacy
guarantees~(\S\ref{appendix:dp}).
%\todo{and demonstrate its superiority to the privacy guarantees of a SOTA
%shaping strategy \cite{cai2014tamaraw} (\S\ref{appendix:tamaraw})}.
%\am{to be revised.}
%\ml{do we want to keep that last point? we claimed it's DP, so we have to have a
%proof. The other I woudn't ref in intro if it's an appendix thing. I think we
%should keep that pointer to where it's relevant in the core of the paper. Or if
%it's so important that it's in the intro, it should be in the core of the
%paper!}
